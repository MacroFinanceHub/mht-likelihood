% !TEX root = ../mhte.tex

\section{Numerical Experiments}
\label{s:experiments}

We have investigated the accuracy of the proposed likelihood approximation by conducting a range of numerical experiments. We discuss the results of three of these experiments here. All three experiments use the default settings for the parameters that control the approximation, unless explicitly stated otherwise. The first two experiments directly compare the explicitly known duration density and likelihood implied by MHT models without shocks to their approximations. The third experiment focuses on a model with shocks, for which the implied duration density is not known in explicit form.

\begin{figure}[t]
\caption{Approximation Error of the Log Likelihood for Various $M$\label{fig:llherr}}
\medskip
\begin{centering}
\begin{tikzpicture}
\tikzstyle{every axis y label}=[yshift=10pt]
\tikzstyle{every axis x label}=[yshift=-5pt]
   \begin{axis}[width=15cm,height=7.5cm,
        name=llherr,
        ylabel={Average absolute error in $\ell_N(\alpha)$},
        xlabel={$M$},
        axis x line=bottom,
        every outer x axis line/.append style={-,color=gray,line width=0.75pt},
        axis y line=left,
        every outer y axis line/.append style={-,color=gray,line width=0.75pt},
        xtick={5,10,15,20,25,30},
        xticklabels={$5$,$10$,$15$,$20$,$25$,$30$},
        ytick={-10,-5,0,5},
        yticklabels={$10^{-10}$,$10^{-5}$,$10^0$,$10^5$},
        major tick length=2.5mm,
        every tick/.append style={line width=0.75pt},
        ymin=-10,
        ymax=5,
       xmin=0,
       xmax=30,
        scaled ticks=false,
        /pgf/number format/precision=2,
        /pgf/number format/set thousands separator={}]
        \addplot[color=blue!75,smooth,mark=*,line width=0.75pt] table[x=M,y=llherr,col sep=comma]{compute/laplace/mhtellherr.csv} ;
         \draw[color=black!75,line width=0.75pt,dashed] (axis cs:0,-5) -- (axis cs:30,-5);
         \draw[color=black!75,line width=0.75pt,dashed] (axis cs:0,0) -- (axis cs:30,0);
\end{axis}
\end{tikzpicture}
\end{centering}

\noindent {\footnotesize Note: This figure is based on the log likelihood $\ell_N(\theta)$ of an MHT model with a Brownian motion latent process and discrete unobserved heterogeneity with three support points for \cites{jem85:kennan} complete strike duration data. It plots the average absolute difference between $\ell_N(\theta)$ and its numerical approximation over 100 randomly drawn parameter values $\alpha$, for a range of values of $M$. The errors are plotted on a logarithmic scale. The parameters are generated using our method of setting starting values for maximum likelihood estimation. This method sets the drift and variance parameters equal to their maximum likelihood estimates for a simple inverse Gaussian model with $\phi(X;\beta)V=1$, which are known in closed form. Starting values for the support points $v_{l}$ of the heterogeneity distribution are generated by exponentiating draws from a standard normal distribution. This ensures that the $v_{l}$ vary in level, but are all approximately of the right scale. All three support points $v_{l}$ receive probability mass $1/3$. The parameter $\beta$ multiplying the covariates is set to zero. For the current experiment, we found that setting the parameters to their final maximum likelihood estimates instead produced almost identical results.}
\end{figure}

The first experiment compares direct computations of the log likelihood function of the mixed inverse Gaussian model using the explicit expression for the density in (\ref{eq:inverseGaussianpdf}) to its numerical approximations as we vary $M$. The log likelihood is calculated on the data set that we use in Section \ref{s:strike}. This ensures that this experiment  provides both a real life test case and a check on the results we present in that section. The data contain 566 complete strike durations. Because the approximation errors are close to unbiased, the error in the log likelihood scales with the root of the sample size.

Figure \ref{fig:llherr} plots the average of the absolute approximation error of the log likelihood, for different values of $M$, over a large set of model parameters randomly generated at the scale of their maximum likelihood estimates. We find that this average absolute error decreases exponentially with $M$; this result is robust across the various parameter values over which the plotted results are averaged.
Consistently with \citet{japr00:rogers}, we see that $M=15$ already provides a decent approximation for most practical purposes. However, because the time required for the calculations grows only linearly in $M$, an extra thousandfold increase in precision can be obtained at a very low computational cost by setting $M=20$ instead. Once $M>20$, other factors, such as rounding errors, become important, and the approximation error levels off. We also find that, with $M=20$, increasing $R$ or decreasing the step size $h$ adds very little to the precision of the inversion. The numerical approximation of the log likelihood takes about 15--20 times as long to calculate as the analytical expression. However, in absolute terms this is still very manageable. For example, it takes only just over 2 seconds to calculate the density for a specification with shocks on a regular desktop computer 100,000 times.\footnote{We used Figure \ref{fig:approxshock}'s specification and MATLAB on a late 2013 iMac with a 2.7 GHz Intel Core i5 processor.}  Consistently with this, the log likelihood can be maximized in under a minute for most model specifications. 

\begin{figure}[t]
\caption{Approximation Error of the Log Inverse Gaussian Density Function\label{fig:proberr}}
\medskip
\begin{centering}
\begin{tikzpicture}
\tikzstyle{every axis y label}=[yshift=10pt]
\tikzstyle{every axis x label}=[yshift=-5pt]
   \begin{axis}[width=15cm,height=7.5cm,
        name=llherr,
        ylabel={Absolute error in $\ln f_{\mathrm{BM}}(t|X)$},
        xlabel={$\ln f_{\mathrm{BM}}(t|X)$},
        axis x line=bottom,
        every outer x axis line/.append style={-,color=gray,line width=0.75pt},
        axis y line=left,
        every outer y axis line/.append style={-,color=gray,line width=0.75pt},
        xtick={-25,-20,-15,-10,-5,0},
        xticklabels={$-25$,$-20$,$-15$,$-10$,$-5$,$0$},
        ytick={-10,-5,0},
        yticklabels={$10^{-10}$,$10^{-5}$,$10^0$},
        major tick length=2.5mm,
        every tick/.append style={line width=0.75pt},
        ymin=-12,
        ymax=2,
       xmin=-28,
       xmax=0,
        scaled ticks=false,
        /pgf/number format/precision=2,
        /pgf/number format/set thousands separator={}]
        \addplot[color=blue!75,only marks,mark=*,mark size=1pt] table[x=lap,y=logerr,col sep=comma]{compute/laplace/mhteproberr.csv} ;
         \draw[color=black!75,line width=0.75pt,dashed] (axis cs:-28,0) -- (axis cs:0,0);
         \draw[color=black!75,line width=0.75pt,dashed] (axis cs:-28,-5) -- (axis cs:0,-5);
         \draw[color=black!75,line width=0.75pt,dashed] (axis cs:-28,-10) -- (axis cs:0,-10);
\end{axis}
\end{tikzpicture}
\end{centering}
\noindent {\footnotesize Note: This figure plots the absolute difference between the log inverse Gaussian density $\ln f_{\mathrm{BM}}(t|X;\theta)$ with parameters $\mu=\sigma^2=\phi(X;\beta)V=1$ and its numerical approximation, on a logarithmic scale, against $\ln f_{\mathrm{BM}}(t|X;\theta)$, for a range of times $t$.}
\end{figure}

The second experiment takes a closer look at the numerical approximation of the density $f_{\mathrm{BM}}$ of a basic inverse Gaussian model with parameters such that $\mu=\sigma^{2}=\phi(X;\beta)V=1$. We only present results for $M=25$, but found very similar results for any $M \geq 15$. For the purpose of maximum likelihood estimation, we care most about the errors in the approximation of the {\em log} density, $\ln f_{\mathrm{BM}}$. Figure \ref{fig:proberr} plots the absolute error of this approximation against the log density itself, on a logarithmic scale. The (log-)linear relation displayed by the graph implies that the absolute error in the approximation of $\ln f_{\mathrm{BM}}(t|X;\theta)$ roughly equals $10^{-11}/f_{\mathrm{BM}}(t|X;\theta)$. Consequently, the approximation error is generally small, but the approximation breaks down when the density gets very small (say, $f_{\mathrm{BM}}(t|X;\theta)<10^{-10}$, or $\ln f_{\mathrm{BM}}(t|X;\theta)<-23$). When estimating the model with maximum likelihood, we can easily avoid this by setting reasonable starting values for the parameters. This ensures that the approximation is sufficiently precise for numerically robust maximum likelihood estimation.

\begin{figure}[t]
\caption{Approximate Probability Density and Histogram of Simulated Values of $\ln T$ for a Specification With Shocks and Heterogeneity\label{fig:histogram}\label{fig:approxshock}}
\medskip
\begin{centering}
\begin{tikzpicture}
\tikzstyle{every axis y label}=[yshift=10pt]
\tikzstyle{every axis x label}=[yshift=-5pt]
   \begin{axis}[width=15cm,height=7.5cm,
        name=llherr,
        ylabel={Density of $\ln T$},
        xlabel={$\ln t$},
        axis x line=bottom,
        every outer x axis line/.append style={-,color=gray,line width=0.75pt},
        axis y line=left,
        every outer y axis line/.append style={-,color=gray,line width=0.75pt},
        major tick length=2.5mm,
        every tick/.append style={line width=0.75pt},
        scaled ticks=false,
        /pgf/number format/precision=2,
        /pgf/number format/set thousands separator={}]
        \addplot[color=green!75,ybar interval,line width=0.50pt] table[x=logtlow,y=hist,col sep=comma]{compute/laplace/mchist.csv} ;
        \addplot[color=blue!75,smooth,line width=0.75pt] table[x=logt,y=invlt,col sep=comma]{compute/laplace/mcinvlap.csv} ;
         \draw[color=black!75,line width=0.75pt,dashed] (axis cs:-28,0) -- (axis cs:0,0);
         \draw[color=black!75,line width=0.75pt,dashed] (axis cs:-28,-5) -- (axis cs:0,-5);
         \draw[color=black!75,line width=0.75pt,dashed] (axis cs:-28,-10) -- (axis cs:0,-10);
   \end{axis}
\end{tikzpicture}
\end{centering}
\noindent {\footnotesize Note: This figure plots the approximate probability density of $\ln T$ (smooth line) and a histogram of $1,000,000$ simulated values of $\ln T$ (bars), for an MHT model in which $\{Y\}$ equals a standard Brownian motion minus an independent compound Poisson process with mean $1/2$ exponential jumps at a rate of one per time unit ($\mu=\sigma=\tau=1$ and $\omega=2$) and the threshold equals $\phi(X;\beta)V=1$ with probability $0.7$ and $\phi(X;\beta)V=5$ with probability $0.3$.}
\end{figure}

The third experiment considers a model with shocks and a heterogeneous threshold. Figure \ref{fig:approxshock} plots the approximate density of $\ln T$ for this model, again using $M=25$. In this case, the true density is not explicitly known, so we compare the approximate density with a fine histogram of many simulated values of $\ln T$. Our approximate density closely tracks the simulated one. This finding is robust across model specifications.
